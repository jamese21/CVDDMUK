\documentclass{article}

\usepackage{url}

\begin{document}
This is a \TeX file.


\section{Introduction}
The term "democracy" is derived from the Ancient Greek term "dēmokratía", which combines "dēmos" (people), and "kratos" (rule), 
literally translating to "rule by the people" \cite{democracy}. This concept, first realised in Athens during the fifth century B.C.E, has since become the
foundation upon which countless political systems and governments have been constructed, aiming to ensure fairness in the distribution of power \cite{natgeo}.

The United Kingdom operates within this framework, employing parliamentary democracy as its system of government \cite{parldem}. In this system, the government of
the UK is answerable to Parliament, which is composed of two bodies: The House of Lords, and the House of Commons. The House of Lords, as the secondary chamber,
is responsible for providing expert scrutiny on government actions and proposed legislation. Its members are appointed rather than elected, and are typically 
recognised experts in their fields. The House of Lords has the power to amend or delay legislation, but it ultimately cannot veto bills passed by the House of 
Commons.

The House of Commons is the primary legislative body of the UK. It is responsible for debating and passing laws, scrutinising the actions of the Prime Minister
and government, and ensuring that the interests of the general population of the UK are represented. It is composed of Members of Parliament (MPs), each of
whom is democratically elected to represent their electoral constituency, and must re-attain their seat in the general elections held at least every five
years \cite{generalelections}. As the more directly democratic chamber that more closely represents the current political climate of the country, the House of Commons plays a crucial
role in shaping policy and maintaining government accountability. The power that a political party holds can be thought of as roughly proportional to the
number of seats it holds in the House of Commons. The party with the most seats is invited to form the government, but, as a majority vote is required in the
House of Commons to pass legislation, not only is it important for the government to maintain the support of a majority of MPs, but the less prominent political
parties also have the agency to influence policy, and decision making is not entirely dominated by the governing and other major parties \cite{divisions}.
This means every single seat in the House of Commons is crucial, and the distribution of these seats among parties is just as important to democratic decision making in the UK
as who the actual winning party is.

A notable example was 1979

There have been 650 seats in the House of Commons, and therefore electoral constituencies, since 2010, and this number has generally hovered around the mid 600s ever
since it was founded with 658 seats in 1801 \cite{numofseats}. This figure came from the 558 seats that already existed in the Parliament 
of Great Britain, and the additional 100 seats that were allocated to Ireland when the two merged to form the modern day House of Commons. While the size of 
the electorate of each constituency is kept roughly equal  (with some exceptions) by the Boundary Commissions, the actual number of constituencies, along
with the criteria considered for boundary allocations, have undergone no significant change since the chamber's inception over 200 years ago \cite{parlcons}.
Given that the UK is a substantially different place now than it was in 1801, with a population that has increased by over fifty million, it seems reasonable to question
if the current system of constituency allocation is still fit for purpose.

Poorly drawn constituencies can have a profound impact on the fairness of political representation. For instance, when a constituency experiences an
overwhelming majority for one party, many votes may 

\section{Objectives}

\section{Literature Review}

\section{Data Methodology}

\section{Partitioning}

\section{Simulation}

\section{Results}

\section{Analysis}

\section{Conclusion}

\section{Evaluation and Critical Appraisal}

\bibliographystyle{plain}
\bibliography{references}

\end{document}